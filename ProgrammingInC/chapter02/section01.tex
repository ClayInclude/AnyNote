\section{程序设计}
{
    从本质上讲,计算机实际上是一种相当死板的装置,它仅仅能够按照人们提供给它的指令运行。
    绝大多数计算机系统所能执行的指令都相当原始,比如给某个数字加1,或者检验某个数字是否等于0等等。
    复合指令比起我们这里给出的例子,也复杂不了多少。
    计算机系统能够执行的基本指令的集合,通常称为该计算机的\emreg{指令集}。

    为了使用计算机解决某个问题,我们必须使用计算机的基本指令描述这个问题的方法。
    所谓计算机\emreg{程序},实际上也就是解决某个具体问题的计算机指令集合。
    按照计算机科学的术语,解决某个具体问题的方法被称为\emreg{算法(algorithm)}。
    举例来说,如果我们想要判断一个数字是奇数还是偶数,那么解决这个问题所需要的指令集合就是一个计算机程序,检验某个数是奇数还是偶数的方法就是算法,再用一段计算机程序来表达这个算法。
    有了算法以后,我们就可以在特定的计算机上,采用该计算机所能接受的指令来实现这个算法。
    这些指令可以是某种特定的计算机编程语言。
}

\endinput
