\section{编译程序}
{
    编译器也是一个程序。从本质上来讲,它和本书中的其他程序是一样的,当然,编译器要比它们复杂得多。
    编译器可以分析使用高级语言编写的程序。,然后把它翻译成特定的计算机系统能够执行的指令。

    要编译一个使用高级语言编写的程序,我们首先要把这个程序输入到计算机\emreg{文件}中。
    虽然在不同的计算机系统中,命名文件的习惯有所不同,但从总体上来说,文件名的选取 还是由使用者决定。
    一般来讲,包含C语言程序的文件以\emcode{.c}两个字母作为文件名的结尾(这一点在更大程度上来说只是一个惯例,而并非强制要求。

    我们通常用文本编辑器把C语言编写的程序输入到计算机系统的文件中。
    使用文本编辑器生成的文件包含了C语言程序的原始形式,因此通常称为\emreg{源文件(Source file)}。
    程序一旦输入到源文件中,我们就可以着手来编译它了。

    在编译的第一阶段,编译器首先检查源程序的每一条语句,看它是否符合语言的语法和词法。
    如果编译器在这个阶段发现了错误,便会将这些错误报告给用户,然后停止运行。

    当程序中所有的语法和语义错误都被改正以后,编译器就会把高级语言编写的源程序翻译为较低级的形式。
    在绝大多数计算机系统上,这些高级语言程序通常首先被翻译为汇编语言程序,这些汇编语言程序完成的功能与高级语言程序相同。

    源程序被翻译为对应的汇编语言程序之后,编译器还需要将这些汇编语言程序翻译成实际的机器指令。
    这个步骤有时需要借助\emreg{汇编器}完成。
    在大多数计算机系统中,编译器通常会自动调用汇编器。

    汇编器读入编译器生成的汇编语言程序,将其翻译为二进制格式的代码,这种代码被称为目标码。
    汇编器将这些生成的目标码保存在目标文件中。

    生成目标文件以后,就可以进行下一个步骤---\emreg{链接(Link)}。
    链接的主要作用是将目标代码转化为具体的计算机系统上实际的可执行程序。
    如果我们在源程序中调用了其他程序的话,那么连接程序就会把这些程序的目标代码和我们的目标代码链接在一起。
    如果我们的程序还使用了系统提供的库函数,这些库函数的代码也会被链接到最后生成的可执行程序中。

    编译和链接一个程序的过程也常常被称为\emreg{构建(building)}。

    链接步骤生成的可执行代码被\emreg{链接器(Linker)}保存在系统的可执行文件中。

    当程序开始运行后,计算机将会按顺序执行程序中的指令。
    如果程序需要从用户那里接受某些数据(也就是输入),系统将暂时挂起程序,以便用户输入。
    又是程序也可以开始等待某些事件的发生,如鼠标点击等。
    程序运行的结果通常输出到一个窗口中,这个窗口被称为\emreg{终端(console)}。
}

\endinput
