\section{理解我们的第一个程序}
{
    程序的第一行如下:

    \begin{codelist}
        \lstinputlisting[firstline = 1, lastline = 1]{./code/example01.c}
    \end{codelist}

    它的作用是告诉编译器\emcode{printf}函数的一些基本信息,在后面的程序中就可以使用这个函数。

    下面的一行程序:

    \begin{codelist}
        \lstinputlisting[firstline = 3, lastline = 3]{./code/example01.c}
    \end{codelist}

    定义了一个名字为\emcode{main}的函数,这个函数的返回值是整型数。
    整型数在C语言中被缩写为\emcode{int}。
    在C语言中,\emcode{main}是一个特殊的函数名字,当程序运行的时候,首先会从\emcode{main}函数开始执行。
    在\emcode{main}后面的一对大括号表明\emcode{main}是一个函数,括号中的关键字\emcode{void}表示\emcode{main}函数不需要任何参数。

    所有被一对大括号包围起来的语句都是\emcode{main}函数的语句。

    \emcode{printf}函数是C语言的一个标准库函数。
    这个函数把传递给它的参数打印在屏幕上。
    字符串的最后两个字符,也就是反斜线和字母\emcode{n},合起来被称为换行符。
    系统在处理输出的时候,如果遇到换行符,就会把后面的输出移到屏幕的下面一行。

    C语言要求所有的语句必须以分号结尾。在\emcode{main}中,调用\emcode{printf}函数的那一行语句后面就有一个分号。

    \emcode{main}函数的最后一个语句如下:

    \begin{codelist}
        \lstinputlisting[firstline = 7, lastline = 7]{./code/example01.c}
    \end{codelist}

    这个语句将结束\emcode{main}函数的运行,并向操作系统返回一个整数0,作为程序的结束状态。
    按照惯例,如果程序返回0值,那就意味着程序一切正常。

    尝试修改程序,让它再打印出一条语句。

    \begin{codelist}
        \lstinputlisting{./code/example02.c}
    \end{codelist}

    如果要输出两行文字,并不一定要调用两次\emcode{printf}函数。

    \begin{codelist}
        \lstinputlisting{./code/example03.c}
    \end{codelist}
}

\endinput
