\section{显示变量的值}
{
    \emcode{printf}函数是本书中使用最频繁的函数,它可以很方便地显示程序运行的结果。
    \emcode{printf}函数不但可以显示简单的文字,也可以显示变量的值以及程序计算的结果。

    使用\emcode{printf}函数,显示50和25相加的结果:

    \begin{codelist}
        \lstinputlisting{./code/example04.c}
    \end{codelist}

    C语言要求函数中使用的变量必须在函数开始的地方声明。
    一个变量的类型表明程序使用这个变量的方式,编译器可以根据变量的类型信息,决定如何存储和访问这个变量,并生成相应的指令。
    整型变量只能用来存放整数。
    带有小数点的数,在C语言中被称为浮点数,或者实数。

    \emcode{printf}函数的第一个参数用来表示需要显示的字符串,但是常常还需要显示程序中某些变量的值。
    为了达到这个目的,我们可以在传递给\emcode{printf}函数的第一个字符串中,嵌入特殊的格式化输出符号。
    格式化输出符号以一个\%开始,后面跟上一个字母。
}

\endinput
