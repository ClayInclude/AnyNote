\chapter{入门}
{
    上个世界70年代初期Dennis Ritchie先生在贝尔实验室发明了C语言,但直到70年代后期,C语言才得以慢慢流传开来。
    因为在那个年代,在贝尔实验室之外不容易获得可用于商业目的的C语言编译器。
    C语言早期的传播部分地归功于Unix操作系统,该操作系统在当时流传的速度丝毫不亚于C语言。
    同C语言一样,Unix操作系统也发源于贝尔实验室。
    这个操作系统 $90\%$ 以上的代码用C语言完成,从而也奠定了C语言作为该操作系统的标准语言的地位。

    IBM PC及其兼容机的巨大成功,使得它们的操作系统MS=DOS很快成为C语言最流行的运行环境。
    随着C语言在多种操作系统上的普及,越来越多的开发商抓住这个机遇,推出了自己的C语言编译器。
    绝大多数编译器遵循的语言标准出自最早的C语言专著《C语言》的一个附录,该书由Brian Kernighan和Dennis Ritchie合著。
    遗憾的是,这个附录并没有完整、严格地定义什么是C语言,因此在C语言的开发过程中,C语言的很多实现细节需要厂商自行决定。

    在上世纪80年代初期,有了对C语言定义标准化的需求。
    \emreg{美国国家标准化协会(The American National Standards Institute, ANSI)}开始着手这方面的工作,并在1983年成立了ANSI C标准化委员会(也称作X3J11)。
    其工作成果于1989年审批通过,并于1990年作为第一部C语言的官方标准(ANSI C)正式发布。

    由于C语言在全世界的广泛应用,\emreg{国际标准化组织(The International Standard Organization, ISO)}也很快参与到标准化工作中来。
    该组织决定采用ANSI的工作成果,并将其编号为ISO/IEC 9899:1990。
    从那以后,C语言还进行性了一些补充修正。
    C语言的最新标准于1999年发布,该标准被称作ANSI C99或者ISO/IEC 9899:1999,本书讲的正是基于该标准的C语言。

    C是一种高级语言,然而它提供了很多途径,使程序员能够在接近底层硬件的水平上对计算机编程。
    这是因为,C语言虽然被设计为一种通用的结构化语言,但在最初构想的时候,设计者也打算使用它进行系统级别的编程。
    正因为如此,C语言具有强大的功能,又不失灵活性。

    本书涵盖了标准C语言的每个特性。
    在介绍这些特性的同时,书中还会附上诠释该特性的一小段完整的程序---用例子帮助读者学习是本书写作过程中遵循的一条原则。

    本书非常强调程序的可读性,因为笔者坚信程序应该于易于阅读的方式编写,无论是供作者还是其他人阅读。
    根据我的经验和常识判断,易读的程序也容易编写、调试和修改。
    除此之外,程序的易读性也是坚持结构化程序风格的一个自然结果。
}

\cleardoublepage

\endinput
