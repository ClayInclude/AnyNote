\chapter{前言}
{
    本书是根据我多年讲授离散数学的经验和兴趣写成的。
    对学生而言,我的目的是为他们提供准确且可读性很强的教材,清晰地介绍并展示离散数学中的概念和技术。
    我的目标是向爱怀疑的学生们展示离散数学的相关性和实用性,希望为学习计算机科学的学生提供一切必须的数学基础,也希望学数学的学生理解重要的数学概念,以及为什么这些对应用来说很重要。

    本教材是为一至两个学期的离散数学入门课程而设计的,适用于数学、计算机科学和工程等各类专业的学生。

    \section{离散数学的目标}
    {
        离散数学课程有多个目标。
        学生不仅要学会一些特定的数学知识并知道怎样应用,更重要的是,这样一门课应该培养学生的数学逻辑思维。
        本书中五个重要主题交织在一起:数学推理、组合分析、离散结构、算法思维、应用与建模。

        \begin{description}
            \item[数学推理]
            {
                学生必须理解数学推理,以便阅读、领会并构造数学论证。
                本书以数理逻辑开篇,在后面证明方法的讨论中,数理逻辑是基础。
                本书描述了构造证明的方法与技巧。
                本书特别强调数学归纳法,不仅给出了这种证明的许多不同类型的实例,还详细的解释了数学归纳法为什么是有效的证明技术。
            }
            \item[组合分析]
            {
                一个重要的解题技巧就是计数或枚举对象。
                本书中,对枚举的讨论从计数的基本技巧着手,重点是用组合分析方法来解决技术问题并分析算法,而不是简单地应用公式。
            }
            \item[离散结构]
            {
                离散数学课程应该教会学生如何处理离散结构,即表示离散对象以及对象之间关系的抽象数学结构。
                离散结构包括集合、置换、关系、图、树和有限状态机等。
            }
            \item[算法思维]
            {
                有些问题可以通过详细说明其算法来求解。
                在清楚地描述算法后,就可以构造一个计算机程序来实现它。
                这一过程中涉及的数学部分包括算法的详细说明、正确性验证以及执行算法所需要的计算机内存和时间的分析等,这些内容在本书中均有介绍。
            }
            \item[应用与建模]
            {
                离散数学几乎在每个可以想象到的研究领域都有应用,本书介绍了其在计算机科学和数据网络中的许多应用,还介绍了在其他各种领域中的应用,如化学、植物学、动物学、语言学、地理学、商业以及因特网等。
                这些均是离散数学的实际而又重要的应用。
                用离散数学来建模是十分重要的问题求解技巧,本书中的一些练习让学生有机会通过自己构造模型掌握这一技巧。
            }
        \end{description}
    }

    \section{本书特色}
    {
        \begin{description}
            \item[容易理解]
            {
                本书对于初学者来说已被实践证明是易读易懂的。
                绝大部分内容不需要读者具备比大学代数更多的数学预备知识。
                需要额外帮助的学生可以在配套网站找到相应工具将数学水平提升到本书的水准。
                本书中少数几个需要参考微积分的地方也已显示注明。
                大多数学生应该很容易理解书中用来表示算法的伪代码。

                每章都是从易于理解和领会的水平开始。
                一旦详细介绍了基本数学概念,就会给出稍难一些的内容以及在其他研究领域中的应用。
            }
            \item[灵活性]
            {
                本书为能灵活使用做了精心设计。
                各章对其前面内容的依赖程度都降到最低。
            }
            \item[写作风格]
            {
                本书的写作风格是直接而又实用的。
                使用准确的数学语言,但没有采用过多的形式化与抽象。
                在数学命题中的记号和词语表达之间做了精心的平衡。
            }
            \item[数学严谨性和准确性]
            {
                本书中所有定义和定理的陈述都十分仔细,这样学生可以欣赏语言的准确性和数学所需的严谨性。
                证明则先是动机再缓慢展开,每一步都经过了详细论证。
                证明中用到的公里及其导出的基本性质在附录中均有显示描述,这呈现给学生一个清晰的概念,即在一个证明中他们能够作何种假设。
                本书解释并大量使用了递归定义。
            }
            \item[实例]
            {
                超过800多个例子用来阐述概念、建立不同主题之间的关联,并介绍应用。
                在大部分例子中,首先提出问题,然后再以适量的细节给出其解。
            }
            \item[应用]
            {
                本书中所含的应用展示了离散数学在解决现实世界中的问题时的实用性。
                本书包含的应用涉及广泛的领域,包括计算机科学、数据网络、心理学、化学、工程学、语言学、生物学和因特网。
            }
            \item[算法]
            {
                离散数学的结论常常要用算法来表述,因此本书每章都介绍一些关键算法。
                这些算法采用文字叙述,同时也采用一种易于理解的结构化伪代码来描述。
                简要分析了书中所有算法的计算复杂性。
            }
            \item[历史资料]
            {
                本书对许多主题的背景做了简要介绍。
                83位数学家和计算机科学家的简短传记以脚注的形式给出。
                这些传记介绍了他们的生活、事业,以及对离散数学做出过重要贡献的科学家的成就。

                此外,脚注还包含了大量历史资料,作为本书正文中历史资料的补充。
            }
            \item[关键术语和结论]
            {
                每章最后列出关键术语和结论。
                关键术语只列出学生必须掌握的那些,而非该章中定义的每个术语。
            }
            \item[练习]
            {
                本书中包含4000多个练习题,涉及大量不同类型的问题。
                不仅提供了足够多的简单练习用于培养基本技能,还提供了大量中等难度的练习和许多具有挑战性的练习。
                练习的叙述清晰而无歧义,并按难易程度进行了分级。
                练习还包含一些特殊的讨论来展开正文中没有涉及的新概念,使得学生能够通过自己的工作来发现新的想法。
            }
            \item[复习题]
            {
                每章最后都有一组复习题。
                设计这些问题是为了帮助学生重点学习该章最重要的概念和技术。
                要回答这些问题,学生必须写出较长的答案,而不是仅作一个计算或一个简答。
            }
            \item[补充练习]
            {
                每章后面都有一组丰富而多样的补充练习。
                这些练习通常比每节后的练习难度更大些。
                补充练习强化该章中的概念,并把不同主题更有效地综合起来。
            }
            \item[计算机课题]
            {
                每章后面还有一组计算机课题。
                大约150个计算机课题将学生在计算和离散数学中所学到的内容联系起来。
            }
            \item[计算和探索]
            {
                每章的最后都有一组计算和探索性的问题。
                完成这些练习需要借助于现有的软件工具,如学生或教师自己编写的程序。
                大部分这些练习为学生提供了通过计算来发现一些新事实或想法的机会。
            }
            \item[写作课题]
            {
                每章后面都有一组写作课题。
                要完成这类课题学生需要参考数学文献。
                有些课题本质上是关于历史的,需要学生查找原始资料。
                有些课题则是通往新内容和新思想的途径。
                所有此类练习是要向学生展示正文中没有深入探讨的想法。
                这些课题把数学概念和写作过程结合起来,以帮助学生面对未来可能的研究领域。
            }
            \item[附录]
            {
                本书有3个附录。
                附录A介绍实数和正整数的公理,并解释如何利用这些公理直接证明事实。
                附录B介绍指数函数和对数函数,复习在课程中常用的一些基本内容。
                附录C则介绍正文中用来描述算法的伪代码。
            }
            \item[推荐读物]
            {
                在附录后还提供了一组针对全书及各章的推荐读物。
                这些推荐读物包括难度不超过本书的书籍、更难些的书籍、阐述性的文章,以及发表离散数学新发现的原始文章。
            }
        \end{description}
    }
}

\cleardoublepage

\endinput
