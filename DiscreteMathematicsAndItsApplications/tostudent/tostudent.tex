\chapter{致学生}
{
    什么是\emreg{离散数学}?
    离散数学是数学中研究离散对象的部分。
    (这里离散意味着由不同的或不相连的元素组成。

    更一般地,每当需要对对象进行计数时,需要研究两个有限(或可数)集合之间的关系时,需要分析涉及有限步骤的过程时,就会用到离散数学。
    离散数学的重要性不断增长的一个关键原因是信息在计算机器中是以离散方式存储和处理的。

    \begin{description}
        \item[为什么要学离散数学]
        {
            学习离散数学有许多重要理由。
            首先,通过这个课程可以培养你的数学素质,即你理解和构造数学论证的能力。
            没有这些技巧,你在学习数学科学时不可能走的太远。
            
            其次,离散数学是学习数学学科中所有高级课程的必由之路。
            离散数学为许多计算机科学课程提供数学基础,这些课程包括数据结构、算法、数据库理论、自动机理论、形式语言、编译理论、计算机安全以及操作系统。
            学生会发现当他们没有从离散数学课程获取适当的数学基础是,要学习这些课程会感到非常困难。

            以离散数学中研究的内容为基础的数学课程包括逻辑、集合论、数论、线性代数、抽象代数、组合学、图论及概率论(其离散部分)。

            此外,离散数学还包含在运筹学(包括许多离散优化技术)、化学、工程学以及生物学等领域问题求解所必需的数学基础。
            在本书中我们将学习上述领域中的一些应用。

            许多学生都感到他们的离散数学入门课程比以前选修过的课程更具挑战性。
            理由就是本课程主要目标是教授数学推理和问题求解,而不是一些零散技巧的集合。
            本书练习的设计就反映了这个目标。
            虽然本书中的大量练习与例题所阐述的类似,但还是有相当比例的练习需要创造性思维。
            这是有意设计的。
            本书中讨论的内容提供了求解这些练习所需的工具,但你的任务是用你自己的创造性成功地使用这些工具。
            本课程的另一个主要目标是学习如何解决那些可能与你以前遇到过的不一样的问题。
            遗憾的是,只学会求解一些特殊类型的练习还不足以保证成功培养在后继课程及职业生涯中所需的问题求解技能。
            本书讨论众多不同的主题,但离散数学是一个极为多样化又涉及广泛的研究领域。
            作者的目标之一是帮助你培养为将来掌握事业中需要的其他知识所必需的技能。
        }
        \item[练习]
        {
            我想就你如何更好地学习离散数学(以及数学和计算机科学中的其他科目)给出一些忠告。
            积极做练习让你收获最大,我建议你尽可能地多做练习。

            做练习的最好方法是首先尝试自己解题,然后再查阅书后的答案。
            你越是尝试自己解题而非被动查阅或照抄解答,你学到的就越多。
        }
    \end{description}
}

\cleardoublepage

\endinput
