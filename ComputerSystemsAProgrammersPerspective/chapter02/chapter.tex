\chapter{信息的表示和处理}
{
    现代计算机存储和处理的信息以二值信号表示。
    这些微不足道的二进制数字,或者称为\emreg{位(bit)},形成了数字革命的基础。
    当构造存储和处理信息的机器时,二进制工作的更好。
    二值信号能够很容易地被表示、存储和传输,例如,可以表示为穿孔卡片上有洞或无洞、导线上的高电压或低电压,或者顺时针或逆时针的磁场。
    对二值信号进行存储和执行计算的电子电路非常简单和可靠,制造商能够在一个单独的硅片上集成数百万甚至数十亿个这样的电路。

    孤立地讲,单个的位不是非常有用。
    然而,当把位组合在一起,再加上某种\emreg{解释(interpretation)},即赋予不同的可能位模式以含意,我们就能够表示任何有限集合的元素。
    比如,使用一个二进制数字系统,我们能够用位组来编码非负数。
    通过使用标准的字符码,我们能够对文档中的字母和符号进行编码。
    在本章中,我们将讨论这两种编码,以及负数表示和实数近似值的编码。

    我们研究三种最重要的数字表示。
    \emreg{无符号(unsigned)}编码基于传统的二进制表示法,表示大于或者等于零的数字表示。
    \emreg{补码(two\'{}s--complement)}编码是表示有符号整数的最常见的方式,有符号整数就是可以为正或者为负的数字。
    \emreg{浮点数(floating--point)}编码是表示实数的科学计数法的以2为基数的版本。
    计算机用这些不同的表示方法实现算术运算,例如加法和乘法,类似于对应的整数和实数运算。

    计算机的表示法是用有限数量的位来对一个数字编码,因此,当结果太大以至不能表示时,某些运算就会\emreg{溢出(overflow)}。
    溢出会导致某些令人吃惊的后果。

    浮点运算有完全不同的数学属性,虽然溢出会产生特殊的值 $+ \infty$ ,但是一组正数的乘积总是正的。
    由于表示的精度有限,浮点运算是不可结合的。
    整数运算和浮点数运算会有不同的数学属性是应为他们处理数字表示有限性的方式不同---整数的表示虽然只能编码一个相对较小的数值范围,但是这种表示是精确的;
    而浮点数虽然可以编码一个较大的数值范围,但是这种表示只是近似的。

    通过研究数字的实际表示,我们能够了解可以表示的值的范围和不同算术运算的属性。
    为了使编写的程序能在全部数值范围内正确工作,而且具有可以跨越不同机器、操作系统和编译器组合的可移植性,了解这种属性是非常重要的。
    后面我们会讲到,大量计算机的安全漏洞都是由于计算机算术运算的微妙细节引发的。
    在早期,当人们碰巧触发了程序漏洞,只会给人们带来一些不便,但是现在,有众多的黑客企图利用他们能找到的任何漏洞,不经过授权就进入他人的系统。
    这就要求程序员有更多的责任和义务,去了解他们的程序如何工作,以及如何被迫产生不良的行为。

    计算机用几种不同的二进制表示形式来编码数值。

    通过直接操作数字的位级表示,我们得到了计中进行算数运算的方式。
    理解这些技术对于理解编译器产生的机器级代码是很重要的,编译器会试图优化算数表达式求值的性能。

    我们对这部分内容的处理是基于一组核心的数学原理。
    从编码的基本定义开始,然后得出一些属性,例如可表示的数字的范围、它们的位级表示以及算术运算的属性。

    \begin{sidenote}[怎样阅读本章]
        本章我们研究在计算机上如何表示数字和其他形式数据的基本属性,以及计算机对这些数据执行操作的属性。
        这就要求我们深入研究数学语言,编写公式和方程式,以及展示重要属性的推导。

        为了帮助你阅读,这部分内容安排如下:首先给出数学形式表示的属性,作为原理。
        然后,用例子和非形式化的讨论来解释这个原理。
        我们建议你反复阅读原理描述和它的示例与讨论,直到你对该属性的说明内容及其重要性有了牢固的直觉。
        对于更加复杂的属性,还会提供推导,其结构看上去将会像一个数学证明。
    \end{sidenote}

    C++编程语言建立在C语言基础之上,它们使用完全相同的数字表示和运算。
    本章中关于C的所有内容对C++都有效。
    另一方面,Java语言创造了一套新的数字表示和运算标准。
    C标准的设计允许多种实现方式,而Java标准在数据的格式和编码上是非常精确具体的。
    本章中多处着重介绍了Java支持的表示和运算。

    \begin{sidenote}[C编程语言的演变]
        前面提到过,C编程语言是贝尔实验室的Dennis Ritchie最早开发出来的,目的是和Unix操作系统一起使用。
        在那个时候,大多数系统程序,例如操作系统,为了访问不同数据类型的低级表示,都必须大量地使用汇编代码。
        比如说,像\emcode{malloc}库函数提供的内存分配功能,用当时的其他高级语言是无法编写的。

        Brain Kernighan和Dennis Ritchie的著作的第1版记录了最初贝尔实验室的C语言版本。
        随着时间的推移,经过多个标准化组的努力,C语言也在不断地演变。
        1989年,美国国家标准协会下的一个工作组推出了ANSI C标准,对最初的贝尔实验室的C语言做了重大修改。
        ANSI C与贝尔实验室的C有了很大的不同,尤其是函数声明的方式。
        Brian Kernighan和Dennis Ritchie在著作的第2版中描述了ANSI C,这本书至今被公认为关于C语言最好的参考手册之一。

        国际标准化组织接替了对C语言进行标准化的任务,在1990年推出了一个几乎和ANSI C一样的版本,称为ISO C90。
        该组织在1999年又对C语言做出了更新,推出ISO C99。
        在这一版中,引入了一些新的数据类型,对使用不符合英语语言字符的文本字符串提供了支持。
        更新的版本2011年得到批准,称为ISO C11,其中再次添加了更多的数据类型和特性。
        最近增加的大多数内容都可以向后兼容,这意味着根据早期标准编写的程序按新标准编译时会有同样的行为。
    \end{sidenote}

    \section{标准头文件}

\endinput

    \section{高级编程语言}
{
    最早的计算机只能理解二进制形式的指令,这些指令通常直接操作内存的某个地址。
    这种形式的指令被称为机器语言。
    第二代计算机编程语言---\emreg{汇编语言(Assembly Language)}允许程序员使用稍微高级一些的指令形式。
    汇编语言可以使用某些字符形式的符号来表示指令和数据。
    因为计算机只能够理解二进制形式的指令,所以人们使用一个特殊的程序---\emreg{汇编器(Assembler)},把汇编语言的程序翻译为具体的机器语言。

    汇编语言使用的符号与计算机的二进制指令实际上是一一对应的。
    正因为如此,汇编语言被看作是低级语言。
    为了使用低级语言编写程序,程序员们必须学习某个特定的计算机系统的指令集。
    由于不同的计算机系统,其指令集常常是不同的。
    因此使用低级语言编写的程序不具备可移植性。
    也就是说,在一种计算机上能运行的程序,如果不进行修改的话,就不能在另一种计算机上运行。

    为了克服低级语言的缺点,人们发明了高级语言。
    \emreg{FORTRAN(FORmula TRANslation)}是第一种高级语言。
    使用Fortran语言,程序员不再需要了解具体计算机系统的结构,Fortran语言提供了更为高级的操作指令。
    一条Fortran语言的指令,或者称为\emreg{表达式(statement)},都对应着很多条机器指令。
    在这点上,Fortran语言与汇编语言很不相同,后者每条指令只能对应一条机器指令。

    高级语言通常有一套特定的语法,这个语法与具体的计算机系统无关。
    因此用高级语言书写的程序,可以独立于具体的计算机系统。
    也就是说,使用高级语言书写的的程序,几乎无需做任何修改,就可以运行在支持该语言的计算机上。

    为了支持某种高级语言,人们需要针对特定计算机系统开发一个特殊的程序,该程序将高级语言编写的程序翻译为特定计算机系统能够理解的机器指令。
    这种计算机程序被称为\emreg{编译器(compiler)}。
}

\endinput

    \section{打开和关闭文件}

\endinput

    \section{期望值和方差}

\endinput

    \section{减少过程调用}

\endinput

    \section{存储设备形成层次结构}
{
    在处理器和一个较大较慢的设备之间插入一个更小更快的存储设备的想法已经成为一个普遍的观念。
    实际上,每个计算机系统中的存储设备都被组织成了一个\emreg{存储器层次结构}。
    在这个层次中,从上至下,设备的访问速度越来越慢,容量越来越大,并且每字节的造价也越来越便宜。
    寄存器文件在层次结构中位于最顶部。

    存储器层次结构的主要思想是上一层的存储器作为低一层存储器的高速缓存。
    在某些具有分布式文件系统的网络系统中,本地磁盘就是存储在其他系统中磁盘上的数据的高速缓存。

    正如可以运用不同的高速缓存的知识来提高程序性能一样,程序员同样可以利用对整个存储器层次结构的理解来提高程序性能。
}

\endinput

}
\cleardoublepage

\endinput
