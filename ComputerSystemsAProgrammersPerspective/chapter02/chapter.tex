\chapter{信息的表示和处理}
{
    现代计算机存储和处理的信息以二值信号表示。
    这些微不足道的二进制数字,或者称为\emreg{位(bit)},形成了数字革命的基础。
    当构造存储和处理信息的机器时,二进制工作的更好。
    二值信号能够很容易地被表示、存储和传输,例如,可以表示为穿孔卡片上有洞或无洞、导线上的高电压或低电压,或者顺时针或逆时针的磁场。
    对二值信号进行存储和执行计算的电子电路非常简单和可靠,制造商能够在一个单独的硅片上集成数百万甚至数十亿个这样的电路。

    孤立地讲,单个的位不是非常有用。
    然而,当把位组合在一起,再加上某种\emreg{解释(interpretation)},即赋予不同的可能位模式以含意,我们就能够表示任何有限集合的元素。
    比如,使用一个二进制数字系统,我们能够用位组来编码非负数。
    通过使用标准的字符码,我们能够对文档中的字母和符号进行编码。
    在本章中,我们将讨论这两种编码,以及负数表示和实数近似值的编码。

    我们研究三种最重要的数字表示。
    \emreg{无符号(unsigned)}编码基于传统的二进制表示法,表示大于或者等于零的数字表示。
    \emreg{补码(two\'{}s--complement)}编码是表示有符号整数的最常见的方式,有符号整数就是可以为正或者为负的数字。
    \emreg{浮点数(floating--point)}编码是表示实数的科学计数法的以2为基数的版本。
    计算机用这些不同的表示方法实现算术运算,例如加法和乘法,类似于对应的整数和实数运算。

    计算机的表示法是用有限数量的位来对一个数字编码,因此,当结果太大以至不能表示时,某些运算就会\emreg{溢出(overflow)}。
    溢出会导致某些令人吃惊的后果。

    浮点运算有完全不同的数学属性,虽然溢出会产生特殊的值 $+ \infty$ ,但是一组正数的乘积总是正的。
    由于表示的精度有限,浮点运算是不可结合的。
    整数运算和浮点数运算会有不同的数学属性是应为他们处理数字表示有限性的方式不同---整数的表示虽然只能编码一个相对较小的数值范围,但是这种表示是精确的;
    而浮点数虽然可以编码一个较大的数值范围,但是这种表示只是近似的。

    通过研究数字的实际表示,我们能够了解可以表示的值的范围和不同算术运算的属性。
    为了使编写的程序能在全部数值范围内正确工作,而且具有可以跨越不同机器、操作系统和编译器组合的可移植性,了解这种属性是非常重要的。
    后面我们会讲到,大量计算机的安全漏洞都是由于计算机算术运算的微妙细节引发的。
    在早期,当人们碰巧触发了程序漏洞,只会给人们带来一些不便,但是现在,有众多的黑客企图利用他们能找到的任何漏洞,不经过授权就进入他人的系统。
    这就要求程序员有更多的责任和义务,去了解他们的程序如何工作,以及如何被迫产生不良的行为。

    计算机用几种不同的二进制表示形式来编码数值。

    通过直接操作数字的位级表示,我们得到了计中进行算数运算的方式。
    理解这些技术对于理解编译器产生的机器级代码是很重要的,编译器会试图优化算数表达式求值的性能。

    我们对这部分内容的处理是基于一组核心的数学原理。
    从编码的基本定义开始,然后得出一些属性,例如可表示的数字的范围、它们的位级表示以及算术运算的属性。

    \begin{sidenote}[怎样阅读本章]
        本章我们研究在计算机上如何表示数字和其他形式数据的基本属性,以及计算机对这些数据执行操作的属性。
        这就要求我们深入研究数学语言,编写公式和方程式,以及展示重要属性的推导。

        为了帮助你阅读,这部分内容安排如下:首先给出数学形式表示的属性,作为原理。
        然后,用例子和非形式化的讨论来解释这个原理。
        我们建议你反复阅读原理描述和它的示例与讨论,直到你对该属性的说明内容及其重要性有了牢固的直觉。
        对于更加复杂的属性,还会提供推导,其结构看上去将会像一个数学证明。
    \end{sidenote}

    C++编程语言建立在C语言基础之上,它们使用完全相同的数字表示和运算。
    本章中关于C的所有内容对C++都有效。
    另一方面,Java语言创造了一套新的数字表示和运算标准。
    C标准的设计允许多种实现方式,而Java标准在数据的格式和编码上是非常精确具体的。
    本章中多处着重介绍了Java支持的表示和运算。

    \begin{sidenote}[C编程语言的演变]
        前面提到过,C编程语言是贝尔实验室的Dennis Ritchie最早开发出来的,目的是和Unix操作系统一起使用。
        在那个时候,大多数系统程序,例如操作系统,为了访问不同数据类型的低级表示,都必须大量地使用汇编代码。
        比如说,像\emcode{malloc}库函数提供的内存分配功能,用当时的其他高级语言是无法编写的。

        Brain Kernighan和Dennis Ritchie的著作的第1版记录了最初贝尔实验室的C语言版本。
        随着时间的推移,经过多个标准化组的努力,C语言也在不断地演变。
        1989年,美国国家标准协会下的一个工作组推出了ANSI C标准,对最初的贝尔实验室的C语言做了重大修改。
        ANSI C与贝尔实验室的C有了很大的不同,尤其是函数声明的方式。
        Brian Kernighan和Dennis Ritchie在著作的第2版中描述了ANSI C,这本书至今被公认为关于C语言最好的参考手册之一。

        国际标准化组织接替了对C语言进行标准化的任务,在1990年推出了一个几乎和ANSI C一样的版本,称为ISO C90。
        该组织在1999年又对C语言做出了更新,推出ISO C99。
        在这一版中,引入了一些新的数据类型,对使用不符合英语语言字符的文本字符串提供了支持。
        更新的版本2011年得到批准,称为ISO C11,其中再次添加了更多的数据类型和特性。
        最近增加的大多数内容都可以向后兼容,这意味着根据早期标准编写的程序按新标准编译时会有同样的行为。
    \end{sidenote}

    \section{信息就是位 $+$ 上下文}
{
    \emreg{hello}程序的生命周期是从一个\emreg{源文件}开始的,即程序员通过编辑器创建并保存的文本文件,文件名是\emreg{hello.c}。
    源程序实际上是一个由值\emcode{0}和\emcode{1}组成的位序列,8个位被组织成一组,称为\emreg{字节}。

    大部分的现代计算机系统都使用ASCII标准来表示文本字符,这种方式实际上就是用一个唯一的单字节大小的整数值来表示每个字符。

    \emreg{hello.c}程序是以字节序列的方式存储在文件中的。
    每个字节都有一个整数值,对应于某些字符。
    每个文本行都是以一个看不见的换行符\emcode{\textbackslash n}来结束的。
    只由ASCII字符构成的文件称为\emreg{文本文件},所有其他文件都称为\emreg{二进制文件}。

    \emreg{hello.c}的表示方法说明了一个基本思想:
    系统中所有的信息---包括磁盘文件、内存中的程序、内存中存放的用户数据以及网络上传送的数据,都是由一串比特表示的。
    区分不同数据对象的唯一方法是我们读到这些数据对象时的上下文。

    \begin{sidenote}[C编程语言的起源]
        C语言是贝尔实验室的Dennis Ritchie于1969年--1973年间创建的。
        \emreg{美国国家标准学会(American National Standards Institute, ANSI)}在1989年颁布了ANSI C的标准,后来C语言的标准化成了\emreg{国际标准化组织(International Standards Organization, ISO)}的责任。
        这些标准定义了C语言和一系列函数库,即所谓的\emreg{C标准库}。
        用Ritchie的话来说,C语言是古怪的、有缺陷的,但同时也是一个巨大的成功。
    
        \begin{description}
            \item[C语言与Unix操作系统关系密切]
            {
                C从一开始就是作为一种用于Unix系统的程序语言开发出来的。
                大部分Unix内核,以及所有支撑工具和函数库都是用C语言编写的。
                20世纪70年代后期到80年代初期,Unix风行于高等院校,许多人开始接触C语言并喜欢上它。
                因为Unix几乎全部是用C编写的,它可以很方便地移植到新的机器上,这种特点为C和Unix赢得了更为广泛的支持。
            }
            \item[C语言小而简单]
            {
                C语言的设计是有一个人而非一个协会掌控的因此这是一个简洁明了、没有什么冗赘的设计。
                K\&R这本书用大量的例子和练习描述了完整的C语言及其标准库,而全书不过261页。
                C元以内的简单使它相对而言易于学习,也易于移植到不同的计算机上。
            }
            \item[C语言是为实践目的设计的]
            {
                C语言是设计用来实现Unix操作系统的。
                后来,其他人发现能够用这门语言毫无障碍的编写他们想要的程序。
            }
        \end{description}
    
        C语言是系统级编程的首选,同时它也非常适用于应用及程序的编写。
        C语言的指针是造成程序员困惑和程序错误的一个常见原因。
        同时,C语言还缺乏对非常有用的抽象的显示支持,例如类、对象和异常。
        像C++和Java这样针对应用及程序的新程序语言解决了这些问题。
    \end{sidenote}
}

\endinput

    \section{赋值和其它类型的语句}

\endinput

    \section{不带输出的有限状态机}

\endinput

    \section{虚拟内存作为内存管理的工具}

\endinput

    \section{高速缓存至关重要}

\endinput

    \section{综合:TINY Web服务器}

\endinput

}
\cleardoublepage

\endinput
