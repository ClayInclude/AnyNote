\section{程序被其他应用程序翻译成不同的格式}
{
    \emreg{hello}程序的生命周期是从一个高级C语言程序开始的,因为这种形式能够被人读懂。
    然而,为了在系统上运行\emreg{hello.c}程序,每条C语句都必须被其他程序转化为一系列低级\emreg{机器语言}指令。
    然后这些指令按照一种称为\emreg{可执行目标程序}的格式打好包,并以二进制磁盘文件的形式存放起来。
    目标程序也称为\emreg{可执行目标文件}。

    在Unix系统上,从源文件到目标文件的转化是由\emreg{编译器驱动程序}完成的:

    \begin{codelist}
        \lstinputlisting[firstline = 1, lastline =1]{./code/shell_collection}
    \end{codelist}

    在这里,GCC编译器驱动程序读取源程序文件\emreg{hello.c},并把它翻译成一个可执行目标文件\emreg{hello}。
    这个翻译过程可以分为四个阶段完成。
    执行这四个阶段的程序(\emreg{预处理器}、\emreg{编译器}、\emreg{汇编器}和\emreg{链接器})一起构成了\emreg{编译系统(compilation system)}。

    \begin{description}
        \item[预处理阶段]
        {
            预处理器(cpp)根据以字符\emcode{\#}开头的命令,修改原始的C程序。
            结果就得到了另一个C程序,通常是以\emcode{.i}作为文件扩展名。
        }
        \item[编译阶段]
        {
            编译器(ccl)将文本文件\emcode{hello.i}翻译成文本文件\emcode{hello.s},它包含一个\emreg{汇编语言程序}。
            该程序包含函数\emcode{main}的定义,如下所示:

            \begin{codelist}
                \lstinputlisting{./code/hello.s}
            \end{codelist}

            定义中每条语句都以一种文本格式描述了一条低级机器语言指令。
            汇编语言是非常有用的,因为它为不同高级语言的不同编译器提供了通用的输出语言。
        }
        \item[汇编阶段]
        {
            接下来,汇编器(as)将\emcode{hello.s}翻译成机器语言指令,把这些指令打包成一种叫做\emreg{可重定位目标程序(relocatable object program)}的格式,并将结果保存在目标文件\emcode{hello.o}中。
            \emcode{hello.o}是一个二进制文件,它包含的字节是函数\emcode{main}的指令编码。
        }
        \item[链接阶段]
        {
            \emcode{hello}程序调用了\emcode{printf}函数,它是每个C编译器都提供的\emreg{标准C库}中的一个函数。
            \emcode{printf}函数存在于一个名为\emcode{printf.o}的单独的预编译好了的目标文件中,而这个文件必须以某种方式合并到\emcode{hello.o}程序中。
            链接器(ld)就负责处理这种合并。
            结果就得到\emcode{hello}文件,它是一个\emreg{可执行目标文件}(或者简称为\emreg{可执行文件}),可以被加载到内存中,由系统执行。
        }
    \end{description}

    \begin{sidenote}[GNU项目]
        GCC是GNU(GUN's Not Unix)项目开发出来的众多有用工具之一。
        GNU项目是1984年由RichardStallman发起的一个免税的慈善项目。
        该项目的目标非常宏大,就是开发出一个完整的类Unix的系统,其源代码能够不受限制地被修改和传播。
        GNU项目已经开发出了一个包含Unix操作系统的所有主要部件的环境,但内核除外,内核是由Linux项目独立发展而来的。
        GNU环境包括EMACS编辑器、GCC编译器、GDB调试器、汇编器、链接器、处理二进制文件的工具以及其他一些部件。
        GCC编译器已经发展到支持许多不同的语言,能够为许多不同的机器生成代码。
        支持的语言包括C、C++、Fortran、Java、Pascal、Objective-C和Ada。

        GNU项目取得了非凡的成绩,但是却常常被忽略。
        现代开放源码运动的思想启源是GNU项目中的自由软件(free software)的概念。
        此处的free为言论自由(free speech)中的自由之意,而非免费啤酒(free beer)中的免费之意。
        而且,Linux如此受欢迎在很大程度上还要归功于GNU工具,它们给Linux内核提供了环境。
    \end{sidenote}
}

\endinput
