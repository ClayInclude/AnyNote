\chapter{计算机系统漫游}
{
    \emreg{计算机系统}是由硬件和系统软件组成的,它们共同工作来运行应用程序。
    虽然系统的具体实现方式随着时间不断变化,但是系统内在的概念却没有改变。
    所有计算机系统都有相似的硬件喝软件组件,它们又执行着相似的功能。

    你将学会一些实践技巧,比如如何避免由计算机表示数字的方式引起奇怪的数字错误。
    你将学会怎样通过一些小窍门来优化自己的C代码,以充分利用现代处理器和存储器系统的设计。
    你将了解编译器是如何实现过程调用的,以及如何利用这些知识来避免缓冲区溢出错误带来的安全漏洞,这些弱点给网络和因特网软件带来了巨大的麻烦。
    你将学会如何识别和避免链接时那些令人讨厌的错误,他们困扰着普通的程序员。
    你将学会如何编写自己的Unix shell、自己的动态存储分配包,甚至于自己的Web服务器。
    你会认识并发带来的希望和陷阱,这个主题随着单个芯片上集成了多个处理器核变得越来越重要。

    尽管\emreg{hello}程序非常简单,但是为了让它实现运行,系统的每个主要组成部分都需要协调工作。
    从某种意义上来说,本书的目的就是要帮助你了解当你在系统上执行\emreg{hello}程序时,系统发生了什么以及为什么会这样。

    \begin{codelist}
        \lstinputlisting{./code/hello.c}
    \end{codelist}

    我们通过跟踪\emcode{hello}程序的生命周期来开始对系统的学习---从它被程序员创建开始,到在系统上运行,输出简单的消息,然后终止。
    我们将沿着这个程序的生命周期,简要地介绍一些逐步出现的关键概念、专业术语和组成部分。

    \section{标准头文件}

\endinput

    \section{高级编程语言}
{
    最早的计算机只能理解二进制形式的指令,这些指令通常直接操作内存的某个地址。
    这种形式的指令被称为机器语言。
    第二代计算机编程语言---\emreg{汇编语言(Assembly Language)}允许程序员使用稍微高级一些的指令形式。
    汇编语言可以使用某些字符形式的符号来表示指令和数据。
    因为计算机只能够理解二进制形式的指令,所以人们使用一个特殊的程序---\emreg{汇编器(Assembler)},把汇编语言的程序翻译为具体的机器语言。

    汇编语言使用的符号与计算机的二进制指令实际上是一一对应的。
    正因为如此,汇编语言被看作是低级语言。
    为了使用低级语言编写程序,程序员们必须学习某个特定的计算机系统的指令集。
    由于不同的计算机系统,其指令集常常是不同的。
    因此使用低级语言编写的程序不具备可移植性。
    也就是说,在一种计算机上能运行的程序,如果不进行修改的话,就不能在另一种计算机上运行。

    为了克服低级语言的缺点,人们发明了高级语言。
    \emreg{FORTRAN(FORmula TRANslation)}是第一种高级语言。
    使用Fortran语言,程序员不再需要了解具体计算机系统的结构,Fortran语言提供了更为高级的操作指令。
    一条Fortran语言的指令,或者称为\emreg{表达式(statement)},都对应着很多条机器指令。
    在这点上,Fortran语言与汇编语言很不相同,后者每条指令只能对应一条机器指令。

    高级语言通常有一套特定的语法,这个语法与具体的计算机系统无关。
    因此用高级语言书写的程序,可以独立于具体的计算机系统。
    也就是说,使用高级语言书写的的程序,几乎无需做任何修改,就可以运行在支持该语言的计算机上。

    为了支持某种高级语言,人们需要针对特定计算机系统开发一个特殊的程序,该程序将高级语言编写的程序翻译为特定计算机系统能够理解的机器指令。
    这种计算机程序被称为\emreg{编译器(compiler)}。
}

\endinput

    \section{打开和关闭文件}

\endinput

    \section{期望值和方差}

\endinput

    \section{减少过程调用}

\endinput

    \section{存储设备形成层次结构}
{
    在处理器和一个较大较慢的设备之间插入一个更小更快的存储设备的想法已经成为一个普遍的观念。
    实际上,每个计算机系统中的存储设备都被组织成了一个\emreg{存储器层次结构}。
    在这个层次中,从上至下,设备的访问速度越来越慢,容量越来越大,并且每字节的造价也越来越便宜。
    寄存器文件在层次结构中位于最顶部。

    存储器层次结构的主要思想是上一层的存储器作为低一层存储器的高速缓存。
    在某些具有分布式文件系统的网络系统中,本地磁盘就是存储在其他系统中磁盘上的数据的高速缓存。

    正如可以运用不同的高速缓存的知识来提高程序性能一样,程序员同样可以利用对整个存储器层次结构的理解来提高程序性能。
}

\endinput

    \section{操作进程的工具}

\endinput

    \section{循环展开}

\endinput

    \section{I/O重定向}

\endinput

    \section{标准I/O}

\endinput

}

\cleardoublepage

\endinput
