\chapter{计算机系统漫游}
{
    \emreg{计算机系统}是由硬件和系统软件组成的,它们共同工作来运行应用程序。
    虽然系统的具体实现方式随着时间不断变化,但是系统内在的概念却没有改变。
    所有计算机系统都有相似的硬件喝软件组件,它们又执行着相似的功能。

    你将学会一些实践技巧,比如如何避免由计算机表示数字的方式引起奇怪的数字错误。
    你将学会怎样通过一些小窍门来优化自己的C代码,以充分利用现代处理器和存储器系统的设计。
    你将了解编译器是如何实现过程调用的,以及如何利用这些知识来避免缓冲区溢出错误带来的安全漏洞,这些弱点给网络和因特网软件带来了巨大的麻烦。
    你将学会如何识别和避免链接时那些令人讨厌的错误,他们困扰着普通的程序员。
    你将学会如何编写自己的Unix shell、自己的动态存储分配包,甚至于自己的Web服务器。
    你会认识并发带来的希望和陷阱,这个主题随着单个芯片上集成了多个处理器核变得越来越重要。

    尽管\emreg{hello}程序非常简单,但是为了让它实现运行,系统的每个主要组成部分都需要协调工作。
    从某种意义上来说,本书的目的就是要帮助你了解当你在系统上执行\emreg{hello}程序时,系统发生了什么以及为什么会这样。

    \begin{codelist}
        \lstinputlisting{./code/hello.c}
    \end{codelist}

    我们通过跟踪\emcode{hello}程序的生命周期来开始对系统的学习---从它被程序员创建开始,到在系统上运行,输出简单的消息,然后终止。
    我们将沿着这个程序的生命周期,简要地介绍一些逐步出现的关键概念、专业术语和组成部分。

    \section{信息就是位 $+$ 上下文}
{
    \emreg{hello}程序的生命周期是从一个\emreg{源文件}开始的,即程序员通过编辑器创建并保存的文本文件,文件名是\emreg{hello.c}。
    源程序实际上是一个由值\emcode{0}和\emcode{1}组成的位序列,8个位被组织成一组,称为\emreg{字节}。

    大部分的现代计算机系统都使用ASCII标准来表示文本字符,这种方式实际上就是用一个唯一的单字节大小的整数值来表示每个字符。

    \emreg{hello.c}程序是以字节序列的方式存储在文件中的。
    每个字节都有一个整数值,对应于某些字符。
    每个文本行都是以一个看不见的换行符\emcode{\textbackslash n}来结束的。
    只由ASCII字符构成的文件称为\emreg{文本文件},所有其他文件都称为\emreg{二进制文件}。

    \emreg{hello.c}的表示方法说明了一个基本思想:
    系统中所有的信息---包括磁盘文件、内存中的程序、内存中存放的用户数据以及网络上传送的数据,都是由一串比特表示的。
    区分不同数据对象的唯一方法是我们读到这些数据对象时的上下文。

    \begin{sidenote}[C编程语言的起源]
        C语言是贝尔实验室的Dennis Ritchie于1969年--1973年间创建的。
        \emreg{美国国家标准学会(American National Standards Institute, ANSI)}在1989年颁布了ANSI C的标准,后来C语言的标准化成了\emreg{国际标准化组织(International Standards Organization, ISO)}的责任。
        这些标准定义了C语言和一系列函数库,即所谓的\emreg{C标准库}。
        用Ritchie的话来说,C语言是古怪的、有缺陷的,但同时也是一个巨大的成功。
    
        \begin{description}
            \item[C语言与Unix操作系统关系密切]
            {
                C从一开始就是作为一种用于Unix系统的程序语言开发出来的。
                大部分Unix内核,以及所有支撑工具和函数库都是用C语言编写的。
                20世纪70年代后期到80年代初期,Unix风行于高等院校,许多人开始接触C语言并喜欢上它。
                因为Unix几乎全部是用C编写的,它可以很方便地移植到新的机器上,这种特点为C和Unix赢得了更为广泛的支持。
            }
            \item[C语言小而简单]
            {
                C语言的设计是有一个人而非一个协会掌控的因此这是一个简洁明了、没有什么冗赘的设计。
                K\&R这本书用大量的例子和练习描述了完整的C语言及其标准库,而全书不过261页。
                C元以内的简单使它相对而言易于学习,也易于移植到不同的计算机上。
            }
            \item[C语言是为实践目的设计的]
            {
                C语言是设计用来实现Unix操作系统的。
                后来,其他人发现能够用这门语言毫无障碍的编写他们想要的程序。
            }
        \end{description}
    
        C语言是系统级编程的首选,同时它也非常适用于应用及程序的编写。
        C语言的指针是造成程序员困惑和程序错误的一个常见原因。
        同时,C语言还缺乏对非常有用的抽象的显示支持,例如类、对象和异常。
        像C++和Java这样针对应用及程序的新程序语言解决了这些问题。
    \end{sidenote}
}

\endinput

    \section{赋值和其它类型的语句}

\endinput

    \section{不带输出的有限状态机}

\endinput

    \section{虚拟内存作为内存管理的工具}

\endinput

    \section{高速缓存至关重要}

\endinput

    \section{综合:TINY Web服务器}

\endinput

    \section{操作进程的工具}

\endinput

    \section{系统之间利用网络通信}

\endinput

    \section{指针和内存地址}

\endinput

    \section{标准I/O}

\endinput

}

\cleardoublepage

\endinput
