\section{小结}
{
    计算机系统是由硬件和软件系统组成的,他们共同协作以运行应用程序。
    计算机内部的信息被表示为一组组的位,它们依据上下文有不同的解释方式。
    程序被其他程序翻译成不同的形式,开始时是ASCII文本,然后被编译器和链接器翻译成二进制可执行文件。

    处理器读取并解释存放在主存里的二进制指令。
    因为计算机花费了大量的时间在内存、I/O设备和CPU寄存器之间复制数据,所以将系统中的存储设备划分成层次结构---CPU寄存器在顶部,接着是多层的硬件高速缓存存储器、DRAM主存和磁盘存储器。
    在层次模型中,位于更高层的存储设备比低层的存储设备要更快,单位比特造价也更高。
    层次结构中较高层次的存储设备可以作为较低层次设备的高速缓存。
    通过理解和运用这种存储层次结构的知识,程序员可以优化C程序的性能。

    操作系统内核是应用程序和硬件之间的媒介。
    它提供三个基本的抽象:

    \begin{enumerate}[(1)]
        \item 文件是对I/O设备的抽象。
        \item 虚拟内存是对主存和磁盘的抽象。
        \item 进程是处理器、主存和I/O设备的抽象。
    \end{enumerate}

    最后,网络提供了计算机系统之间通信的手段。
    从特殊系统的角度来看,网络就是一种I/O设备。
}

\endinput
